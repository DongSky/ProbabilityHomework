\chapter{Bins and Balls - Poisson Approximation}
	
	\section{Poisson Distribution}
	\textbf{Problem}\\
	Assume that for each $n$,there are Bernoulli random variables $B^n_1,\cdots,B^n_n$ with indicators $X^n_i$.Let $Y_n=\sum_{i=1}^{n}$,suppose that\\
	(a)$\lim\limits_{n \rightarrow \infty}E[Y_n]=\lambda$,and\\
	(b)For any $k$,$\lim\limits_{1 \leq i_1 < \cdots < i_k \leq n}Pr(\bigcap_{r=1}^{k}B^n_{i_r})=\frac{\lambda^k}{k!}$.\\
	Prove that $\lim\limits_{n \rightarrow \infty}Pois(\lambda)$,i.e. $\lim\limits_{n \rightarrow \infty}Pr(Y_n=k)=\frac{e^{-\lambda}\lambda^k}{k!}$ for any $k$.\\\\
	\textbf{Solution}\\
	\begin{proof}
		Since $Y_n=\sum_{i=1}^{n}X_i^n$,so $Y_n$ satisfies Binomial distribution.Suppose $Y_n\sim Bin(n,p)$.\\
	Since $E[Y_n]=\sum_{i=1}^{n}Y_i \cdot p=\lambda$,so $p=\frac{\lambda}{n}$.
	Therefore
	\begin{equation*}
		\begin{split}
			\lim\limits_{n \rightarrow \infty}Pr(Y_n=k)
			&= \lim\limits_{n \rightarrow \infty} \binom{n}{k} p^k (1-p)^{n-k}\\
			&= \lim\limits_{n \rightarrow \infty} \frac{n!}{k!(n-k)!} \left(\frac{\lambda}{n} \right)^k \left(1-\frac{\lambda}{n} \right)^{n-k}\\
			&= \lim\limits_{n \rightarrow \infty} \underbrace{\frac{n!}{n^k (n-k)!}}_{\rightarrow 1} \frac{\lambda^k}{k!} \underbrace{\left(1-\frac{\lambda}{n} \right)^{n-k}}_{\rightarrow e^{-\lambda}} \\
			&= \frac{\lambda^k e^{-\lambda}}{k!} 
		\end{split}
	\end{equation*}
	Such that the proof ends.\\
	I also noticed that this problem is the case of weak independence,the following proof goes when $Y_n>0$,but as for $Y_n=k$,I basically have no idea how to figure it out.I tried Inclusion-exclusion principle and Taylor's expansion,but it turned out not working.\\
	Since $\lim\limits_{n\rightarrow \infty}E[Y_n]=\lim\limits_{n\rightarrow \infty}E \left[\sum_{i=1}^{n}X_i^n \right]=\lim\limits_{n\rightarrow \infty}\sum_{i=1}^{n}E[X_i^n]=\lambda$,and $X_i^n$ satisfies Bernoulli distribution,so $\lim\limits_{n\rightarrow \infty}Pr(X_i^n)=\lambda$.\\
	According to $inclusion-exclusion$ principle
	\begin{equation*}
		\begin{split}
			Pr(Y_n=k)
			&= Pr \left(\sum_{i=1}^{n}X_i^n=k \right)\\
			&= \sum_{k=1}^{n}(-1)^{k-1} \sum_{1 \leq i_1 \le i_2 \le \cdots \le i_k \leq n}^{}Pr \left(\bigcap_{r=1}^{k}X_{i_r}^n \right)\\
			&= Pr \left(\sum_{i=1}^{n}X_i^n \right) - \sum_{1 \leq i_j \le i_k \leq n}^{}Pr(X_i^n \cap X_j^n) + \cdots +(-1)^{n-1}\sum_{1 \leq i_1 \le i_2 \le \cdots \le i_k \leq n}^{} Pr \left(\bigcap_{r=1}^{n}X_{i_r}^n \right)\\
			&= \lambda - \sum_{k=1}^{n}(-1)^{k-1}\frac{\lambda^k}{k!}
		\end{split}
	\end{equation*}
	Besides,according to $Taylor's$ expansion,we have
	\begin{equation*}
		\lim_{n \rightarrow \infty} \left(\lambda - \frac{\lambda^2}{2}+\frac{\lambda^3}{3!}- \cdots + (-1)^{n-1}\frac{\lambda^{n-1}}{(n-1)!} \right)=e^{-\lambda}
	\end{equation*}
	Such that we can simply get $\lim\limits_{n\rightarrow \infty}Pr(Y_n=k)=\frac{e^{-\lambda}\lambda^k}{k!}$.
	\end{proof}
	
	\section{Poisson Distribution 2}
	\textbf{Problem}\\
	Define the Poisson experiment: $n$ bins, each is filled with independent $Poisson(m/n)$ balls. Let $E$ be the event that no bins are empty, and $X$ be the total number of balls filled with. Prove that $Pr(E|X = k)$ increases with $k$.\\\\
	\textbf{Solution}\\
	\begin{proof}
		If we set variable $Y$ as the number of empty bins, so we can get that
		\[
		Pr(E|X = k) = Pr(Y = 0|X = k) = 1 - \sum_{i = 1}^{n - 1}Pr(Y = i | X = k)
		\]
		Because $\{Y = i | X = k + 1\}$ can be derived by two situations:
		\begin{itemize}
			\item By $\{Y = i | X = k\}$ and the next ball fall into the $n - k$ bins who are not empty.
			\item By $\{Y = i + 1|X= k\}$ and the next ball fall into any one bin of the $i + 1$ bins who are empty.
		\end{itemize}
		So we can get that
		\[
		Pr(Y = i|X = k+1) = Pr(Y=i|X = k)\frac{n - i}{n} + Pr(Y = i| X = k)\frac{i + 1}{n}.
		\]
		Therefore
		\begin{equation*}
			\begin{split}
				Pr(E | X = k + 1)
				&= Pr(Y = 0 | X = k + 1)\\
				&= 1 - \sum_{i = 1}^{n - 1}Pr(Y = i | X = k + 1)\\
				&= 1 - \sum_{i = 1}^{n - 1}Pr(Y = i|X = k)\frac{n - i}{n} - \sum_{i = 1}^{n - i}Pr(Y = i + 1|X = k)\frac{i + 1}{n}\\
				&= 1 - \sum_{i = 1}^{n - 1}Pr(Y = i | X = k)\frac{n - i}{n} - \sum_{i = 2}^{n}Pr(Y = i | X = k)\frac{i}{n}\\
				&=1 - \frac{n - 1}{n}Pr(Y = 1 | X = k) - \sum_{i = 2}^{n - 1}Pr(Y = i|X = k)\frac{n - i}{n} - \sum_{i = 2}^{n}Pr(Y = i|X = k)\frac{i}{n}\\
				&=1 - \frac{n - 1}{n}Pr(Y = 1|X = k) - \sum_{i = 2}^{n - 1}\left(\frac{n - i}{n}+\frac{i}{n}\right)Pr(Y = i| X = k)\\
				&=1 - \sum_{i = 2}^{n - 1}Pr(Y = i|X = k) - Pr(Y = 1|X = k) + \frac{1}{n}Pr(Y = 1|X = k)\\
				&= 1 - \sum_{i = 1}^{n - 1}Pr(Y = i | X = k) + \frac{1}{n}Pr(Y = 1|X = k)\\
				&= Pr(E|X = k) + \frac{1}{n}Pr(Y = 1 | X = k)\\
				&> Pr(E|X = k).
			\end{split}
		\end{equation*}
		Such that the proof ends.
	\end{proof}
	\textbf{Proof using Poisson approximation}
	
		\begin{proof}
			The number of balls in the bins satisfies Poisson distribution,so
	\begin{equation*}
		Pr(k=0)=\frac{e^{-\lambda \lambda^k}}{k!}=e^{-\frac{m}{n}}
	\end{equation*}
	So the probability of no bins are empty is
	\begin{equation*}
		Pr(\epsilon)=\left(1-e^{-\frac{m}{n}}\right)^n=e^{-ne^{-\frac{m}{n}}}
	\end{equation*}
	According to Poisson Approximation
	\begin{equation*}
		Pr(\epsilon|X=k)=e^{-ne^{-\frac{k}{n}}}
	\end{equation*}
	It's clear this function goes monotonically with $k$.
		\end{proof}

	
	\section{Poisson Distribution 3}
	\textbf{Problem}\\
	Follow the Poisson experiment and the notation in the last question. Assume that $m=n\ln n +cn$. Rigorously prove that $Pr(\epsilon|X=m+\sqrt{2m\ln m})-Pr(\epsilon|X=m-\sqrt{2m\ln m}) \rightarrow 0$ as $n \rightarrow \infty$.\\\\
	\textbf{Solution}\\
	\begin{proof}
		From the book we know that
	\begin{equation*}
		Pr(\epsilon|X=m+\sqrt{2m\ln m})-Pr(\epsilon|X=m-\sqrt{2m\ln m}) \geq |Pr(\epsilon||X-m|\leq \sqrt{2m\ln m})-Pr(\epsilon|X=m)|
	\end{equation*}
	And because
	\begin{equation*}
		|Pr(\epsilon||X-m|\leq \sqrt{2m\ln m})-Pr(\epsilon|X=m)|=o(1)
	\end{equation*}
	So the limit is $o(1)$.
	\end{proof}
	
	\section{Agent and Resource}
	\textbf{Problem}\\
	The following problem models a simple distributed system wherein agents contend for resources but back off in the face of contention. Balls represent agents, and bins represent resources.\\
	The system evolves over rounds. Every round, balls are thrown independently and uniformly at random into $n$ bins. Any ball that lands in a bin by itself is served and removed from consideration. The remaining balls are thrown again in the next round. We begin with n balls in the first round, and we will finish when every ball is served.
	\begin{itemize}
		\item If there are $b$ balls at the start of a round, what is the expected number of balls at the start of the next round?
		\item Suppose that every round the number of balls served was exactly the expected number of balls to be served. Show that all the balls would be served in $O(ln ln n)$ rounds. (Hint: If $x_j$ is the expected number of balls left after j rounds, show and use that $x_{j+1} \le x^2_j/n$.)
	\end{itemize}
	\textbf{Solution}\\
	\begin{itemize}
		\item
		Consider the probability that a bin has 1 ball,denote it as $X_i$,then the probability satisfies $binomial$ distribution,so
		\begin{equation*}
		Pr(X_i)=\binom{b}{1} \frac{1}{n} \left(1-\frac{1}{n} \right)^{b-1}
		\end{equation*}
		Such that for $n$ bins,the expected number of balls of $n$ bins after 1 round is
		\begin{equation*}
		E\left[\sum_{i=1}^{n} X_i\right]= \sum_{i=1}^{n} E(X_i)= b \left(1-\frac{1}{n}\right)^{b-1}
		\end{equation*}
		Therefore the expected number of balls left is $b-b \left(1-\frac{1}{n}\right)^{b-1}$.
		\item
		\begin{proof}
			From above we have
		\begin{equation*}
		\begin{split}
		x_{j+1} 
		&= x_j-x_j \left(1-\frac{1}{n}\right)^{x_j-1}\\
		&\leq x_j-x_j \left(1-\frac{x_j-1}{n}\right)\\
		&= \frac{x_j^2}{n}
		\end{split}
		\end{equation*}
		So
		\begin{equation*}
		\begin{split}
		x_1 &\leq \frac{x_0^2}{n}\\
		x_2 \leq \frac{x_1^2}{n} &\leq \frac{x_0^4}{n^3}\\
		\cdots\\
		x_j &\leq \frac{x_0^{2^j}}{n^{2^j-1}}
		\end{split}
		\end{equation*}
		Note that $x_0=n$,we simply let $x_{j-1}=1$,which means after $j$ rounds,there will be no balls left.
		Since $x_j \leq \frac{x_0^{2^j}}{n^{2^j-1}}$,so
		\begin{equation*}
		\begin{split}
		2^j \leq 2j\ln x_0-\ln x_j +1\\
		j \leq \ln \left(2j\ln x_0 - \ln x_j +1\right)\\
		j \leq \ln \ln x_0+\ln2j
		\end{split}
		\end{equation*}
		Since $x_{j-1}=1,x_0=n$,we can simply get $j \leq \ln \ln n$,which means $j=O(\ln \ln n)$.
		\end{proof}
	\end{itemize}

	\section{Coin}
	\textbf{Problem}\\
	Do Bernoulli experiment for 20 trials, using a new 1-Yuan coin.Write down the results in a string $s_1 s_2 \cdots s_{20}$, where $s_i$ is 1 if the $i$-th trial gets Head,and otherwise is 0.
	\\
	\textbf{Solution}\\
	The result of tossing a coin 20 times is:\\11101000111010101000.
	

